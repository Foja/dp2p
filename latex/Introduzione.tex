\section{Introduzione}
\subsection{DP2P}
Lo scopo del progetto è quello di realizzare un'architettura P2P ibrida per la condivisione di file con directory distribuita.
Un sistema peer-to-peer è un sistema distribuito in cui ogni nodo ha identiche capacità e responsabilità e le funzionalità di un peer sono totalmente indipendenti da un server.
\subsection{Specifiche}
La struttura P2P ha principalmente le seguenti caratteristiche:
\begin{itemize}
\item controllo decentralizzato;
\item adattabilità;
\item capacità autonome di gestione e controllo di risorse e servizi;
\end{itemize}
Il sistema P2P viene definito ibrido quando:
\begin{itemize}
\item la fase di scambio dei file è peer-to-peer;
\item la fase di boot utilizza qualche server;
\item la fase di ricerca dei file usa un peer particolare, chiamato superpeer;
\end{itemize}
L'architettura è perciò organizzata in questo modo:
\begin{itemize}
\item ogni peer è associato ad un superpeer;
\item i superpeer tengono traccia dei file condivisi dai peer ad esso associati e sono collegati tra loro tramite una rete  “overlay”; 
\item il server centralizzato di bootstrap tiene traccia dei superpeer presenti.
\end{itemize}
\subsection{Obiettivi}
Il software realizzato permette:
\begin{itemize}
\item l'ingresso e l'uscita di un nodo peer nella rete;
\item la registrazione e l'uscita dei superpeer e la gestione della rete di overlay;
\item la visualizzazione sul peer della lista dei file locali condivisi;
\item la ricerca di un determinato file posseduto da un altro nodo della rete, tramite interrogazione del superpeer;
\item il download di un file da un peer.
\end{itemize}
\subsection{Esecuzione}
L'esecuzione di un sistema P2P consiste solitamente di tre fasi distinte:
\begin{enumerate}
\item FASE DI BOOT: inizializzazione del sistema, entrata e registrazione di ogni nodo nella rete;
\item FASE DI LOOKUP: ricerca di un file,interazione peer-superpeer e successivo inoltro della richiesta nella rete di overlay;
\item FASE DI DOWNLOAD: trasferimento del file richiesto. 
\end{enumerate}


